\documentclass[10pt]{article}

% The automated optical recognition software used to digitize resume
% information works best with fonts that do not have serifs. This
% command uses a sans serif font throughout. Uncomment both lines (or at
% least the second) to restore a Roman font (i.e., a font with serifs).
%\usepackage{times}
%\renewcommand{\familydefault}{\sfdefault}

% This is a helpful package that puts math inside length specifications
\usepackage{calc}
\usepackage{comment}
\usepackage[T1]{fontenc}
\usepackage[utf8]{inputenc}
\usepackage{lmodern}
\usepackage{tabu}
%\usepackage{hyperref}

% Layout: Puts the section titles on left side of page
\reversemarginpar

%% Use these lines for letter-sized paper
\usepackage[paper=letterpaper,
	%includefoot, % Uncomment to put page number above margin
	marginparwidth=1.2in,     % Length of section titles
	marginparsep=.05in,       % Space between titles and text
	margin=1in,               % 1 inch margins
	includemp]{geometry}

%% Use these lines for A4-sized paper
%\usepackage[paper=a4paper,
%            %includefoot, % Uncomment to put page number above margin
%            marginparwidth=30.5mm,    % Length of section titles
%            marginparsep=1.5mm,       % Space between titles and text
%            margin=25mm,              % 25mm margins
%            includemp]{geometry}

\setlength{\parindent}{0in}

\usepackage[shortlabels,inline]{enumitem}
\usepackage{fancyhdr,lastpage}
\pagestyle{empty}

% Finally, give us PDF bookmarks
\usepackage{color,hyperref}
\definecolor{darkblue}{rgb}{0.0,0.0,0.6}
\hypersetup{colorlinks,breaklinks, linkcolor=darkblue,urlcolor=darkblue, anchorcolor=darkblue,citecolor=darkblue}

%%%%%%%%%%%%%%%%%%%%%%%%%%% Helper Commands %%%%%%%%%%%%%%%%%%%%%%%%%%%%

% The title (name) with a horizontal rule under it
% (optional argument typesets an object right-justified across from name
%  as well)
%
% Usage: \makeheading{name}
%        OR
%        \makeheading[right_object]{name}
%
% Place at top of document. It should be the first thing.
% If ``right_object'' is provided in the square-braced optional
% argument, it will be right justified on the same line as ``name'' at
% the top of the CV. For example:
%
%       \makeheading[\emph{Curriculum vitae}]{Your Name}
%
% will put an emphasized ``Curriculum vitae'' at the top of the document
% as a title. Likewise, a picture could be included:
%
%   \makeheading[\includegraphics[height=1.5in]{my_picutre}]{Your Name}
%
% the picture will be flush right across from the name.
\newcommand{\makeheading}[2][]%
{\hspace*{-\marginparsep minus \marginparwidth}%
	\begin{minipage}[t]{\textwidth+\marginparwidth+\marginparsep}%
		{\large \bfseries #2 \hfill #1}\\[-0.15\baselineskip]%
		\rule{\columnwidth}{1pt}%
	\end{minipage}}

% The section headings
%
% Usage: \section{section name}
\renewcommand{\section}[1]{\pagebreak[3]%
	\hyphenpenalty=10000%
	\vspace{1.3\baselineskip}%
	\phantomsection\addcontentsline{toc}{section}{#1}%
	\noindent\llap{\scshape\smash{\parbox[t]{\marginparwidth}{\raggedright #1}}}%
	\vspace{-\baselineskip}\par}

% An itemize-style list with lots of space between items
\newenvironment{outerlist}[1][\enskip\textbullet]%
{\begin{itemize}[#1,leftmargin=*]}{\end{itemize}%
	\vspace{-.6\baselineskip}}

% An environment IDENTICAL to outerlist that has better pre-list spacing
% when used as the first thing in a \section
\newenvironment{lonelist}[1][\enskip\textbullet]%
{\begin{list}{#1}{%
			\setlength{\partopsep}{0pt}%
			\setlength{\topsep}{0pt}}}
		{\end{list}\vspace{-.6\baselineskip}}

% An itemize-style list with little space between items
\newenvironment{innerlist}[1][\enskip\textbullet]%
{\begin{itemize}[#1,leftmargin=*,parsep=0pt,itemsep=0pt,topsep=0pt,partopsep=0pt]}
		{\end{itemize}}

% An environment IDENTICAL to innerlist that has better pre-list spacing
% when used as the first thing in a \section
\newenvironment{loneinnerlist}[1][\enskip\textbullet]%
{\begin{itemize}[#1,leftmargin=*,parsep=0pt,itemsep=0pt,topsep=0pt,partopsep=0pt]}
		{\end{itemize}\vspace{-.6\baselineskip}}

% To add some paragraph space between lines.
% This also tells LaTeX to preferably break a page on one of these gaps
% if there is a needed pagebreak nearby.
\newcommand{\blankline}{\quad\pagebreak[3]}
\newcommand{\halfblankline}{\quad\vspace{-0.5\baselineskip}\pagebreak[3]}

% Uses hyperref to link DOI
\newcommand\doilink[1]{\href{http://dx.doi.org/#1}{#1}}
\newcommand\doi[1]{doi:\doilink{#1}}

% For \url{SOME_URL}, links SOME_URL to the url SOME_URL
\providecommand*\url[1]{\href{#1}{\texttt{#1}}}

% For \email{ADDRESS}, links ADDRESS to the url mailto:ADDRESS
\providecommand*\email[1]{\href{mailto:#1}{\texttt{#1}}}

\newenvironment{detaillist}{\begin{itemize}[label={},leftmargin=*,nolistsep,labelindent=10pt]}{\end{itemize}}

\usepackage{totcount}
\usepackage{ifmtarg}
\newtotcounter{publicationsTot}
\newcounter{publications}
\makeatletter
\newcommand{\ifnonempty}[2]{\@ifnotmtarg{#1}{#2}}
\makeatother
\newenvironment{publication}[5][]{\addtocounter{publicationsTot}{1}\addtocounter{publications}{-1}\setlength{\parskip}{0pt}%
  \begin{itemize}[leftmargin=*,labelindent=0pt,nolistsep]%
    \setlength{\parskip}{0pt}%
  \item[(\thepublications)]{#2}.\linebreak[0]\emph{#3}. #4 #5\ifnonempty{#1}{, #1}.%
  \end{itemize}}{}

%%%%%%%%%%%%%%%%%%%%%%%%% Begin CV Document %%%%%%%%%%%%%%%%%%%%%%%%%%%%

\begin{document}
\setcounter{publications}{\totvalue{publicationsTot}}
\addtocounter{publications}{1}
\sloppy
\makeheading{\huge{Wil Thomason}}

\section{Contact Information}

% NOTE: Mind where the & separators and \\ breaks are in the following
%       table.
%
% ALSO: \rcollength is the width of the right column of the table
%       (adjust it to your liking; default is 1.85in).
%
\newlength{\rcollength}\setlength{\rcollength}{1.4in}%
%
\begin{tabu} to 1.0\textwidth {X[l]X[r]}
	Cornell University             & 804.591.7318                          \\
	Department of Computer Science & \email{wbthomason@cs.cornell.edu}     \\
    343 Campus Road                & \url{https://www.cs.cornell.edu/~wil} \\
	Ithaca, NY 14853                                                       \\
\end{tabu}

\section{Objective}

I am broadly interested in algorithms for robotics, with a focus on \textbf{planning algorithms} and
\textbf{integrated task and motion planning}. My work studies efficient, principled approaches to
solving realistic robot problems with both logical and geometric complexity. I am also interested in
fusing classical planning methods with machine learning to enable more robust, capable robots, as
well as algorithms for efficient planning in constrained domains, programming languages research,
and robot autonomy/decision-making. In all my work, I care about finding solutions that are both
soundly based in theory and practical for real-world use.

\section{Research Interests}

\textbf{Robotics}: integrated task and motion planning, constrained planning, multi-agent
coordination\\
\textbf{Machine Learning}: learning to plan, combining ML with classical planning\\
\textbf{Programming Languages}: type theory, effect systems, compilers

\section{Education}

\href{https://www.cs.cornell.edu}{\textbf{Cornell University}},
Ithaca, NY\hfill{June 2019 -- Present}
\begin{innerlist}
	\item[] Ph.D., Computer Science
\end{innerlist}\bigskip

\href{https://www.cs.cornell.edu}{\textbf{Cornell University}},
Ithaca, NY\hfill{August 2015 -- June 2019}
\begin{innerlist}
	\item[] MS, Computer Science
	\item[] \textbf{Relevant Courses:}
	\emph{CS 6820}: Advanced Algorithms (Fall 2015), \emph{CS 6786}: Advanced Machine Learning
	(Spring 2016), \emph{CS 6110}: Advanced Programming Languages (Spring 2016), \emph{MAE 6790}:
	Intelligent Sensor Planning and Control (Fall 2016), \emph{CS 6751}: Intro to Mobile
	Manipulation (Spring 2017), \emph{MAE 6770}: Formal Methods for Robotics (Fall 2017)
\end{innerlist}\bigskip

\href{https://www.virginia.edu}{\textbf{University of Virginia}},
Charlottesville, VA\hfill{August 2012 -- May 2015}
\begin{innerlist}
	\item[] BS, Computer Science, Mathematics
	\item[] \textbf{Relevant Courses:}
	\emph{CS 6161}: Graduate Design and Analysis of Algorithms (Fall 2014), \emph{CS 6610}: Graduate
	Programming Languages (Fall 2014)
\end{innerlist}

\section{Awards}

\textbf{Outstanding Teaching Assistant Award}\\\textit{Cornell University Department of Computer Science}, May of 2017
\begin{innerlist}
	\item[] Awarded for my work as a TA for the new Foundations of Robotics course in Fall 2016,
	taught by Ross Knepper.
\end{innerlist}\bigskip

\textbf{NDSEG Fellow} \hfill \textit{American Society for Engineering Education}, April of 2017
\begin{innerlist}
	\item[] The National Defense Science and Engineering Graduate (NDSEG) Fellowship is a highly
	competitive, portable fellowship that is awarded to U.S. citizens and nationals who intend to
	pursue a doctoral degree in one of fifteen supported disciplines. NDSEG confers high honors
	upon its recipients, and allows them to attend whichever U.S. institution they choose.
	(\url{https://ndseg.asee.org/})
\end{innerlist}\bigskip

\textbf{NSF GRFP Fellow} \hfill \textit{The National Science Foundation}, March of 2017
\begin{innerlist}
	\item[] The NSF Graduate Research Fellowship Program recognizes and supports outstanding
	graduate students in NSF-supported science, technology, engineering, and mathematics
	disciplines who are pursuing research-based Master's and doctoral degrees at accredited United
	States institutions. (\url{https://www.nsfgrfp.org/})
\end{innerlist}\bigskip

\textbf{Outstanding Teaching Assistant Award}\\\textit{Cornell University Department of Computer Science}, May of 2016
\begin{innerlist}
	\item[] Awarded for my work as a TA for the Intro to Python course (CS 1110) in Fall 2015,
	taught by Walker White.
\end{innerlist}\bigskip

\textbf{NSF GRFP Honorable Mention} \hfill \textit{The National Science Foundation}, March of 2016
\begin{innerlist}
	\item[] The NSF Graduate Research Fellowship Program recognizes and supports outstanding
	graduate students in NSF-supported science, technology, engineering, and mathematics
	disciplines who are pursuing research-based Master's and doctoral degrees at accredited United
	States institutions.(\url{https://www.nsfgrfp.org/})
\end{innerlist}\bigskip

\textbf{Louis T. Rader Outstanding Education Undergraduate Student Award} \hfill \textit{University of Virginia Department of Computer Science}, May of 2015
\begin{innerlist}
	\item[] Awarded for my work for several years as a TA for multiple courses in the computer
	science department at UVA.
\end{innerlist}

\section{Peer-Reviewed Conference Publications}
\begin{publication}[in submission]{Ensuring Safety and Progress for Independent Multi-Robot Teams in Shared
  Space}{\textbf{Wil Thomason}, Claire Liang, Elizabeth Ricci, and Soham Sankaran}{WAFR}{2020}
\end{publication}

% \textbf{Ensuring Safety and Progress for Independent Multi-Robot Teams in Shared Space}
% \emph{WAFR (in submission)} (2020)
% \begin{innerlist}
% 	\item[] \emph{Wil Thomason, Claire Liang, Elizabeth Ricci, Soham Sankaran}
% 	\item[] We present a framework for guaranteeing safety and progress for all robots operating in a
% 	shared environment containing multiple independent multi-robot teams. Our technique uses a
% 	partition of the freespace into \emph{flow regions} and \emph{open space} to avoid deadlock in
% 	narrow passageways and provide a mechanism for dynamically redistributing robot density in the
% 	environment, and constructs a simple priority-based protocol for requesting access to a spot in
% 	the freespace that guarantees all robots can always eventually reach their goals.
% \end{innerlist}\bigskip

\begin{publication}[in submission]{Automatic Distributed Multi-Agent Coordination of Single-Agent
  Robot Controllers}{\textbf{Wil Thomason}, Abhishek Anand, Greg Morrisett, Ross Knepper}{IROS}{2020}
\end{publication}

% \emph{IROS RA-L (in submission)} (2020)
% \begin{innerlist}
% 	\item[] \emph{Wil Thomason, Abhishek Anand, Greg Morrisett, Ross Knepper.}
% 	\item[] We present a framework for automatically transforming single-agent robot controllers
% 	into distributed multi-agent controllers with guaranteed safety and progress under lightweight
% 	assumptions. The core of this framework is a simple protocol that maintains two invariants:
% 	(a) all executed actions are non-colliding, and (b) there is always a valid plan to reverse
% 	the system back to an earlier state.
% \end{innerlist}\bigskip

\begin{publication}{A Unified Sampling-Based Approach to Integrated Task and Motion
  Planning}{\textbf{Wil Thomason}, Ross Knepper}{ISRR}{2019}
\end{publication}

% \textbf{A Unified Sampling-Based Approach to Integrated Task and Motion Planning.}
% \emph{ISRR} (2019)
% \begin{innerlist}
% 	\item[] \emph{Wil Thomason, Ross Knepper}.
% 	\item[] We contribute a novel approach to integrated task and motion planning that adapts any
% 	off-the-shelf motion planning algorithm to automatically solve \emph{task and motion planning
% 		problems}. We construct a state space embedding the full symbolic and continuous state of a
% 	planning problem into a single highly factorable space. Combined with a continuous semantics
% 	for Boolean predicates, which we use to derive navigation functions to task-level salient
% 	regions of the continuous space, we can efficiently solve complex manipulation problems with
% 	few domain-specific assumptions.
% \end{innerlist}\bigskip

\begin{publication}{Social Momentum: A Framework for Legible Navigation in Dynamic Multi-Agent
  Environments}{Christoforos Mavrogiannis, \textbf{Wil Thomason}, Ross Knepper}{HRI}{2018}
\end{publication}

% \textbf{Social Momentum: A Framework for Legible Navigation in Dynamic Multi-Agent Environments.}
% \emph{HRI} (2018)
% \begin{innerlist}
% 	\item[] \emph{Christoforos Mavrogiannis, Wil Thomason, Ross Knepper.}
% 	\item[] We design a planning framework that aims at generating motion that clearly communicates
% 	an agent’s intended collision avoidance strategy rather than its destination. Our framework
% 	estimates the most likely intended avoidance protocols of others based on their past
% 	behaviors, superimposes them, and generates an expressive and socially compliant robot action
% 	that reinforces the expectations of others regarding these avoidance protocols.
% \end{innerlist}\bigskip

\textbf{Zero-Shot Learning for Unfamiliar Gesture Recognition.}
\emph{ISER} (2016)
\begin{innerlist}
	\item[] \emph{Wil Thomason, Ross Knepper.}
	\item[] We explore a method for achieving increased understanding of complex, situated
	communications by leveraging coordinated natural language, gesture, and context. These three
	problems have largely been treated separately, but unified consideration of them can yield
	gains in comprehension.
\end{innerlist}\bigskip

\textbf{An Accurate Real-Time RFID-Based Location System.}
\emph{IJRFITA} (2016)
\begin{innerlist}
	\item[] \emph{Kirti Chawla, Christopher McFarland, Gabriel Robins, Wil Thomason.}
	\item[] We outline an RFID-based object localization framework addressing challenges of
	performance and applicability and propose the use of Received Signal Strength (RSS) to model
	the behavior of radio signals decaying over distance in an orientation-agnostic manner to
	simultaneously locate a number of stationary and mobile objects.
\end{innerlist}

\section{Journal Publications}
\begin{publication}[in submission]{Automatic Distributed Multi-Agent Coordination of Single-Agent
  Robot Controllers}{\textbf{Wil Thomason}, Abhishek Anand, Greg Morrisett, Ross Knepper}{Robotics
and Automation Letters.}{February 2020}
\end{publication}

\section{Presentations}

``A Unified Sampling-Based Approach to Integrated Task and Motion Planning.''\\
\emph{ISRR 2019}\hfill October 09, 2019
\begin{innerlist}
	\item[] \url{https://h2t-projects.webarchiv.kit.edu/Projects/\\ISRR2019/index.php/program-and-talks/program/}
\end{innerlist}\bigskip

``A Flexible Sampling-Based Approach to Task and Motion Planning.''\\
\emph{RSS 2019 Workshop on Robust Task and Motion Planning}\hfill
June 23, 2019
\begin{innerlist}
	\item[] \url{http://dyalab.mines.edu/2019/rss-workshop/thomason.pdf}
\end{innerlist}\bigskip

``Which comes first, the task plan or the motion plan?.'' (joint with Ross Knepper)\\
\emph{RSS 2018 Workshop on Exhibition and Benchmarking of Task and Motion Planners}\hfill
June 30, 2018
\begin{innerlist}
	\item[] \url{http://www.neil.dantam.name/2018/rss-tmp-workshop/}
\end{innerlist}\bigskip

``Exploiting Heterogeneity in Robot Teams Through a Formalism of Capabilities.''\\
\emph{RSS 2018 Workshop on Heterogeneity and Diversity for Resilience in Multi-Robot Systems}\hfill
July 15, 2017
\begin{innerlist}
	\item[] \url{https://www.cs.cornell.edu/\textasciitilde{}wil/papers/\\rss2017\_workshop\_heterogeneous\_coordination.pdf}
\end{innerlist}\bigskip

``Toward Contextual Grounding of Unfamiliar Gestures for Human-Robot Interaction.''\\
\emph{FG 2017: First International Workshop on Adaptive Shot Learning for Gesture Understanding and Production}\hfill
May 30, 2017
\begin{innerlist}
	\item[] \url{https://www.cs.cornell.edu/\textasciitilde{}wil/papers/asl4gup2017\_unfamiliargestures.pdf}
\end{innerlist}\bigskip

``Recognizing Unfamiliar Gestures for Human-Robot Interaction through Zero Shot Learning.''\\
\emph{ISER 2016}\hfill
October 6th, 2016
\begin{innerlist}
	\item[] \url{http://www.iser2016.org/program/}
\end{innerlist}\bigskip

``Recognizing Unfamiliar Gestures for Human-Robot Interaction through Zero-Shot Learning.''\\
\emph{2nd Workshop on Model Learning for Human-Robot Communication, RSS 2016}\hfill
June 19th, 2016
\begin{innerlist}
	\item[] \url{http://www.ece.rochester.edu/projects/rail/mlhrc2016/}
\end{innerlist}

\section{Service}

I have served as a reviewer for ICRA (2016, 2019, 2020), IROS (2019), MRS (2019), RO-MAN (2016), RSS
(2019), WAFR (2018), AURO, and SIMPAR (2018).
I have also served as a reviewer and mentor for \hyperlink{https://blackinai.github.io/}{Black in
	AI}, a workshop at NeurIPS that works to promote and increase the
presence of Black people in the field of artificial intelligence (2017--2019).

\section{Outreach}

\textbf{Expanding Your Horizons}\hfill Spring 2016 -- Spring 2019\\
Taught middle school girls from the community about programming
\\\url{https://www.eyh.cornell.edu/}
\begin{itemize}
	\setlength\itemsep{0em}
	\item Created interface for introducing the concept of programming with state machines
	\item Wrote ROS software to run state machine code on a Kuka youBot
\end{itemize}
\bigskip\textbf{UVa High School Programming Contest}\hfill Spring 2014 -- Spring 2015\\
Helped plan and run the biggest programming competition for high school students in the mid-Atlantic region.
\\\url{http://acm.cs.virginia.edu/hspc.php}
\begin{itemize}
	\setlength\itemsep{0em}
	\item Created contest problems
	\item Planned contest logistics
	\item Helped run contest
\end{itemize}
\bigskip\textbf{UVa CS Education Week}\hfill Winter 2014 -- Winter 2015\\
Visited area schools to teach workshops on introductory programming with JKarel.

\section{Research Experience}
\textbf{Graduate Research Assistant} \hfill {January 2020 -- Present}\\
\emph{Cornell University, Department of Computer Science}
\begin{innerlist}
	\item[] I am continuing my work on integrated task and motion planning with Professor Hadas
	Kress-Gazit.
\end{innerlist}\bigskip

\textbf{Graduate Research Assistant} \hfill {August 2015 -- December 2019}\\
\emph{Cornell University, Department of Computer Science}
\begin{innerlist}
	\item[] Work with Professor Ross Knepper on problems in the domains of integrated
	task and motion planning, human-robot interaction and multi-agent planning.
	See publications co-authored with Professor Knepper.
\end{innerlist}\bigskip

\textbf{Undergraduate Research Assistant} \hfill {August 2014 -- July 2015}\\
\emph{The University of Virginia, Department of Computer Science}
\begin{innerlist}
	\item[] Work with Professor Westley Weimer on automatic software functionality
	transplantation. We developed an algorithm based on analyzing differences in test suite
	performance to identify software modules responsible for specific functionality and a
	combination of function dependency tracing and extraction with the GenProg program repair tool
	to perform transplants.
\end{innerlist}\bigskip

\textbf{Undergraduate Research Assistant} \hfill {January 2013 -- May 2014}\\
\emph{The University of Virginia, Department of Computer Science}
\begin{innerlist}
	\item[] Work with Professor Gabriel Robins on efficient, scalable,
	real-time localization of objects in 3D space using passive RFID tags.
	See publication ``An Accurate Real-Time RFID-Based Location System''.
\end{innerlist}

\section{Teaching Experience}

\textbf{Graduate Teaching Assistant: CS 4750} \hfill Fall 2016 \& Fall 2017\\
\emph{Cornell University}
\begin{innerlist}
	\item[] TA for Foundations of Robotics: A new course designed to introduce students to the
	knowledge they need to conduct research in robotics. I helped to create the course syllabus,
	textbook, and software for course assignments, and was also responsible for giving weekly
	office hours, grading, and assisting with lecturing.
\end{innerlist}\bigskip

\textbf{Graduate Teaching Assistant: CS 1110} \hfill Fall 2015\\
\emph{Cornell University}
\begin{innerlist}
	\item[] Head TA for Cornell's introductory computer science course. Responsible for coordinating
	TA staff, giving review lectures, supervising lab sessions, grading, and giving weekly office
	hours.
\end{innerlist}\bigskip

\textbf{Instructor: Introduction to Robotics} \hfill Spring 2015\\
\emph{University of Virginia}
\begin{innerlist}
	\item[] I designed and taught a 1-credit course introducing undergraduate students to core
	topics in robotics. As a part of the course, students built and programmed their own quadrotor
	robots and learned about basic kinematics, control, perception, and learning.
\end{innerlist}\bigskip

\textbf{Undergraduate Teaching Assistant: CS 4610} \hfill Spring 2015\\
\emph{University of Virginia}
\begin{innerlist}
	\item[] TA for UVA's undergraduate programming languages course.
\end{innerlist}\bigskip

\textbf{Undergraduate Teaching Assistant: CS 4710} \hfill Spring 2015\\
\emph{University of Virginia}
\begin{innerlist}
	\item[] TA for UVA's undergraduate artificial intelligence course.
\end{innerlist}\bigskip

\textbf{Undergraduate Teaching Assistant: CS 4414} \hfill Spring 2014\\
\emph{University of Virginia}
\begin{innerlist}
	\item[] TA for UVA's undergraduate operating systems course. Helped create course content using the Rust programming language as well as an automatic grading server for the class.
\end{innerlist}\bigskip

\textbf{Undergraduate Teaching Assistant: CS 2150} \hfill Fall 2013 -- Spring 2015\\
\emph{University of Virginia}
\begin{innerlist}
	\item[] TA for UVA's undergraduate data structures and C++ programming course.
\end{innerlist}

\section{Industry Experience}

\textbf{Software Engineering Intern at Fluencia}\hfill May 2015 -- August 2015\\
Worked on adding voice understanding for speech practice exercises
\begin{itemize}
	\setlength\itemsep{0em}
	\item Improved voice understanding software
	\item Integrated into language learning website
	\item Created test and development framework for voice exercises
\end{itemize}\bigskip

\textbf{Software Development Engineer Intern at Microsoft}\hfill May 2014 -- August 2014\\
Microsoft Accounts Client Team
\begin{itemize}
	\setlength\itemsep{0em}
	\item Implemented and tested cryptographic operations and network protocol for forthcoming
	      feature in Microsoft Accounts Android app. The implemented feature enables more convenient and
	      more secure sign-in for Microsoft accounts via mobile devices.
	\item Coordinated the work of several other interns working on related components of the same
	      project to ensure successful integration and delivery of the fully-functional feature ahead of
	      schedule.
\end{itemize}\bigskip

\textbf{Software Development Engineer Intern at Microsoft}\hfill May 2013 -- August 2013\\
Xbox LIVE Cloud Security Team
\begin{itemize}
	\setlength\itemsep{0em}
	\item Designed, implemented, and shipped a RESTful web service capable of logging and auditing
	      security records in Xbox LIVE in real time. Also designed, created, and tuned associated
	      database and procedures. Service is currently in use in the Xbox LIVE network.
	\item Spearheaded and completed a total rewrite of an important development library – starting
	      with old, cobbled together library used across principal components of the Xbox LIVE network,
	      redesigned and reimplemented the entire library to provide a faster and easier to use
	      interface to the same core functionality.
\end{itemize}

\section{Technical Skills}

\textbf{Programming Languages:} C++, Python, Rust, C, CL, Haskell, OCaml, etc.\\
\textbf{Technologies:} Linux, ROS, OMPL, PyTorch, TensorFlow, Git, CUDA, etc.

\section{Languages}

English \emph{(Fluency: Native speaker)}\\
German \emph{(Fluency: Working)}
\end{document}

%%%%%%%%%%%%%%%%%%%%%%%%%% End CV Document %%%%%%%%%%%%%%%%%%%%%%%%%%%%%

%----------------------------------------------------------------------%
% The following is copyright and licensing information for
% redistribution of this LaTeX source code; it also includes a liability
% statement. If this source code is not being redistributed to others,
% it may be omitted. It has no effect on the function of the above code.
%----------------------------------------------------------------------%
% Copyright (c) 2007, 2008, 2009, 2010, 2011 by Theodore P. Pavlic
%
% Unless otherwise expressly stated, this work is licensed under the
% Creative Commons Attribution-Noncommercial 3.0 United States License. To
% view a copy of this license, visit
% http://creativecommons.org/licenses/by-nc/3.0/us/ or send a letter to
% Creative Commons, 171 Second Street, Suite 300, San Francisco,
% California, 94105, USA.
%
% THE SOFTWARE IS PROVIDED "AS IS", WITHOUT WARRANTY OF ANY KIND, EXPRESS
% OR IMPLIED, INCLUDING BUT NOT LIMITED TO THE WARRANTIES OF
% MERCHANTABILITY, FITNESS FOR A PARTICULAR PURPOSE AND NONINFRINGEMENT.
% IN NO EVENT SHALL THE AUTHORS OR COPYRIGHT HOLDERS BE LIABLE FOR ANY
% CLAIM, DAMAGES OR OTHER LIABILITY, WHETHER IN AN ACTION OF CONTRACT,
% TORT OR OTHERWISE, ARISING FROM, OUT OF OR IN CONNECTION WITH THE
% SOFTWARE OR THE USE OR OTHER DEALINGS IN THE SOFTWARE.
%----------------------------------------------------------------------%
