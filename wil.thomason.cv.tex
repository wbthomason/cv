\documentclass[10pt]{article}

% The automated optical recognition software used to digitize resume
% information works best with fonts that do not have serifs. This
% command uses a sans serif font throughout. Uncomment both lines (or at
% least the second) to restore a Roman font (i.e., a font with serifs).
%\usepackage{times}
%\renewcommand{\familydefault}{\sfdefault}

\usepackage[hyphens]{url}
% This is a helpful package that puts math inside length specifications
\usepackage{calc}
\usepackage{comment}
\usepackage[T1]{fontenc}
\usepackage[utf8]{inputenc}
\usepackage{lmodern}
\usepackage{tabu}

% Layout: Puts the section titles on left side of page
\reversemarginpar

%% Use these lines for letter-sized paper
\usepackage[paper=letterpaper,
	%includefoot, % Uncomment to put page number above margin
	marginparwidth=1.2in,     % Length of section titles
	marginparsep=0.05in,       % Space between titles and text
	margin=0.7in,               % 1 inch margins
	includemp]{geometry}

%% Use these lines for A4-sized paper
%\usepackage[paper=a4paper,
%            %includefoot, % Uncomment to put page number above margin
%            marginparwidth=30.5mm,    % Length of section titles
%            marginparsep=1.5mm,       % Space between titles and text
%            margin=25mm,              % 25mm margins
%            includemp]{geometry}

\setlength{\parindent}{0in}

\usepackage[shortlabels,inline]{enumitem}
\usepackage{fancyhdr,lastpage}
\pagestyle{empty}

\usepackage{totcount}
\usepackage{ifmtarg}

\usepackage[final]{microtype}
% Finally, give us PDF bookmarks
\usepackage{color}
\usepackage{hyperref}
\definecolor{darkblue}{rgb}{0.0,0.0,0.6}
\hypersetup{colorlinks,breaklinks,linkcolor=darkblue,urlcolor=darkblue, anchorcolor=darkblue,citecolor=darkblue}

%%%%%%%%%%%%%%%%%%%%%%%%%%% Helper Commands %%%%%%%%%%%%%%%%%%%%%%%%%%%%

% The title (name) with a horizontal rule under it
% (optional argument typesets an object right-justified across from name
%  as well)
%
% Usage: \makeheading{name}
%        OR
%        \makeheading[right_object]{name}
%
% Place at top of document. It should be the first thing.
% If ``right_object'' is provided in the square-braced optional
% argument, it will be right justified on the same line as ``name'' at
% the top of the CV. For example:
%
%       \makeheading[\emph{Curriculum vitae}]{Your Name}
%
% will put an emphasized ``Curriculum vitae'' at the top of the document
% as a title. Likewise, a picture could be included:
%
%   \makeheading[\includegraphics[height=1.5in]{my_picutre}]{Your Name}
%
% the picture will be flush right across from the name.
\newcommand{\makeheading}[2][]%
{\hspace*{-\marginparsep minus \marginparwidth}%
	\begin{minipage}[t]{\textwidth+\marginparwidth+\marginparsep}%
		{\large \bfseries #2 \hfill #1}\\[-0.15\baselineskip]%
		\rule{\columnwidth}{1pt}%
	\end{minipage}}

% The section headings
%
% Usage: \section{section name}
\renewcommand{\section}[1]{\pagebreak[3]%
	\hyphenpenalty=10000%
	\vspace{1.3\baselineskip}%
	\phantomsection\addcontentsline{toc}{section}{#1}%
	\noindent\llap{\scshape\smash{\parbox[t]{\marginparwidth}{\raggedright #1}}}%
	\vspace{-\baselineskip}\par}

% An itemize-style list with lots of space between items
\newenvironment{outerlist}[1][\enskip\textbullet]%
{\begin{itemize}[#1,leftmargin=*]}{\end{itemize}%
	\vspace{-.6\baselineskip}}

% An environment IDENTICAL to outerlist that has better pre-list spacing
% when used as the first thing in a \section
\newenvironment{lonelist}[1][\enskip\textbullet]%
{\begin{list}{#1}{%
			\setlength{\partopsep}{0pt}%
			\setlength{\topsep}{0pt}}}
		{\end{list}\vspace{-.6\baselineskip}}

% An itemize-style list with little space between items
\newenvironment{innerlist}[1][\enskip\textbullet]%
{\begin{itemize}[#1,leftmargin=*,parsep=0pt,itemsep=0pt,topsep=0pt,partopsep=0pt]}
		{\end{itemize}}

% An environment IDENTICAL to innerlist that has better pre-list spacing
% when used as the first thing in a \section
\newenvironment{loneinnerlist}[1][\enskip\textbullet]%
{\begin{itemize}[#1,leftmargin=*,parsep=0pt,itemsep=0pt,topsep=0pt,partopsep=0pt]}
		{\end{itemize}\vspace{-.6\baselineskip}}

% Fake "subsections" through bolded list items with good spacing
\newenvironment{loneboldlist}%
{\begin{description}[leftmargin=*,parsep=0pt,itemsep=0pt,topsep=0pt,partopsep=0pt]}
		{\end{description}\vspace{-.6\baselineskip}}

% To add some paragraph space between lines.
% This also tells LaTeX to preferably break a page on one of these gaps
% if there is a needed pagebreak nearby.
\newcommand{\blankline}{\quad\pagebreak[3]}
\newcommand{\halfblankline}{\quad\vspace{-0.5\baselineskip}\pagebreak[3]}

% Uses hyperref to link DOI
\newcommand\doilink[1]{\href{http://dx.doi.org/#1}{#1}}
\newcommand\doi[1]{doi:\doilink{#1}}

% For \url{SOME_URL}, links SOME_URL to the url SOME_URL
\providecommand*\url[1]{\href{#1}{\texttt{#1}}}

% For \email{ADDRESS}, links ADDRESS to the url mailto:ADDRESS
\providecommand*\email[1]{\href{mailto:#1}{\texttt{#1}}}

% Environment for compact, bullet-free lists of items
\newenvironment{detaillist}{\begin{itemize}[label={},leftmargin=*,nolistsep,labelindent=10pt]}{\end{itemize}}

% Environments for conference and journal publications
\newtotcounter{cpublicationsTot}
\newcounter{cpublications}
\newtotcounter{jpublicationsTot}
\newcounter{jpublications}
\makeatletter
\newcommand{\ifnonempty}[2]{\@ifnotmtarg{#1}{#2}}
\makeatother
\newenvironment{publication}[6][]{\setlength{\parskip}{0pt}%
	\begin{itemize}[leftmargin=0.5cm,labelindent=0pt,nolistsep]%
		\item[#2.]\emph{#3}.~{#4}. #5 #6\ifnonempty{#1}{, #1}.}{\end{itemize}}
\newenvironment{confpub}[5][]{\addtocounter{cpublicationsTot}{1}\addtocounter{cpublications}{-1}
	\begin{publication}[#1]{\thecpublications}{#2}{#3}{#4}{#5}\end{publication}}{}
\newenvironment{journpub}[5][]{\addtocounter{jpublicationsTot}{1}\addtocounter{jpublications}{-1}
	\begin{publication}[#1]{\thejpublications}{#2}{#3}{#4}{#5}\end{publication}}{}

% Environment for awards
\newenvironment{award}[3]{\setlength{\parskip}{0pt}%
	\begin{itemize}[label={},leftmargin=0cm,labelindent=0pt,nolistsep]%
		\item\textbf{#1}\hfill\quad{#3}\linebreak[1]~\emph{#2}}{\end{itemize}\medskip}

% Environment for symposia/consortia
\newenvironment{symposium}[3]{\setlength{\parskip}{0pt}%
	\begin{itemize}[label={},leftmargin=0cm,labelindent=0pt,nolistsep]%
		\item\textbf{#1}\hfill\quad{#3}\linebreak[1]~#2}{\end{itemize}\medskip}

% Environment for courses
\newenvironment{class}[4]{\setlength{\parskip}{0pt}%
	\begin{itemize}[label={},leftmargin=0cm,labelindent=0pt,nolistsep]%
		\item\textbf{#1}~\emph{({#2})}~\hfill\emph{#3}, #4}{\end{itemize}\medskip}

% Environment for work experience items
          \newenvironment{job}[3]{\setlength{\parskip}{0pt}%
	\begin{itemize}[label={},leftmargin=0cm,labelindent=0pt,nolistsep]%
		\item\textbf{#1}~\hfill\emph{#2}\\{#3}.}{\end{itemize}\medskip}
%%%%%%%%%%%%%%%%%%%%%%%%% Begin CV Document %%%%%%%%%%%%%%%%%%%%%%%%%%%%

\begin{document}
\setcounter{cpublications}{\totvalue{cpublicationsTot}}
\addtocounter{cpublications}{1}
\setcounter{jpublications}{\totvalue{jpublicationsTot}}
\addtocounter{jpublications}{1}
\sloppy
\makeheading{\huge{Wil Thomason}}

\section{Contact Information}

% NOTE: Mind where the & separators and \\ breaks are in the following
%       table.
%
% ALSO: \rcollength is the width of the right column of the table
%       (adjust it to your liking; default is 1.85in).
%
\newlength{\rcollength}\setlength{\rcollength}{1.4in}%
%
\hspace{-\tabcolsep}\noindent
\begin{tabu} to 1.03\textwidth {X[l]X[r]}
	Cornell University             & 804.591.7318                          \\
	Department of Computer Science & \email{wbthomason@cs.cornell.edu}     \\
	343 Campus Road                & \url{https://www.cs.cornell.edu/~wil} \\
	Ithaca, NY 14853                                                       \\
\end{tabu}

% \section{Objective}
%
% I am broadly interested in algorithms for robotics, with a focus on \textbf{planning algorithms} and
% \textbf{integrated task and motion planning}. My work studies efficient, principled approaches to
% solving realistic robot problems with both logical and geometric complexity. I am also interested in
% fusing classical planning methods with machine learning to enable more robust, capable robots, as
% well as algorithms for efficient planning in constrained domains, programming languages research,
% and robot autonomy/decision-making. In all my work, I care about finding solutions that are both
% soundly based in theory and practical for real-world use.

\section{Research Interests}

Robotics, integrated task and motion planning, ML for planning, constrained planning, formal methods
for robotics, synthesis, motion planning, multi-agent coordination
% \textbf{Programming Languages}: type theory, effect systems, compilers

\section{Education}
\textbf{Cornell University}, Ithaca, NY\hfill{Present}
\begin{detaillist}
	\item\emph{Ph.D. in Computer Science}. Advisor: Hadas Kress-Gazit.
\end{detaillist}\medskip

\textbf{Cornell University}, Ithaca, NY\hfill{June 2019}
\begin{detaillist}
	\item\emph{MS in Computer Science}. Advisor: Ross A. Knepper.
	% \item \textbf{Relevant Courses:}
	% \emph{CS 6820}: Advanced Algorithms (Fall 2015), \emph{CS 6786}: Advanced Machine Learning
	% (Spring 2016), \emph{CS 6110}: Advanced Programming Languages (Spring 2016), \emph{MAE 6790}:
	% Intelligent Sensor Planning and Control (Fall 2016), \emph{CS 6751}: Intro to Mobile
	% Manipulation (Spring 2017), \emph{MAE 6770}: Formal Methods for Robotics (Fall 2017)
\end{detaillist}\medskip

\textbf{University of Virginia}, Charlottesville, VA\hfill{August 2012 -- May 2015}
\begin{detaillist}
	\item\emph{BS (with high distinction) in Computer Science and Mathematics}
	% \item \textbf{Relevant Courses:}
	% \emph{CS 6161}: Graduate Design and Analysis of Algorithms (Fall 2014), \emph{CS 6610}: Graduate
	% Programming Languages (Fall 2014)
\end{detaillist}

\section{Awards}
\begin{award}{Outstanding Teaching Assistant Award}{Cornell University Department of Computer
		Science}{May 2017}
	% Awarded for my work as a TA for the new Foundations of Robotics course in Fall 2016,
	% taught by Ross Knepper.
\end{award}

\begin{award}{NDSEG Fellow}{American Society for Engineering Education}{April 2017}
	% The National Defense Science and Engineering Graduate (NDSEG) Fellowship is a highly
	% competitive, portable fellowship that is awarded to U.S. citizens and nationals who intend to
	% pursue a doctoral degree in one of fifteen supported disciplines. NDSEG confers high honors
	% upon its recipients, and allows them to attend whichever U.S. institution they choose.
	% (\url{https://ndseg.asee.org/})
\end{award}

\begin{award}{NSF GRFP Fellow}{The National Science Foundation}{March 2017}
	% The NSF Graduate Research Fellowship Program recognizes and supports outstanding
	% graduate students in NSF-supported science, technology, engineering, and mathematics
	% disciplines who are pursuing research-based Master's and doctoral degrees at accredited United
	% States institutions. (\url{https://www.nsfgrfp.org/})
\end{award}

\begin{award}{Outstanding Teaching Assistant Award}{Cornell University Department of Computer
		Science}{May 2016}
	%  Awarded for my work as a TA for the Intro to Python course (CS 1110) in Fall 2015,
	% taught by Walker White.
\end{award}

\begin{award}{NSF GRFP Honorable Mention}{The National Science Foundation}{March 2016}
	%  The NSF Graduate Research Fellowship Program recognizes and supports outstanding
	% 	graduate students in NSF-supported science, technology, engineering, and mathematics
	% 	disciplines who are pursuing research-based Master's and doctoral degrees at accredited United
	% 	States institutions.(\url{https://www.nsfgrfp.org/})
\end{award}

\begin{award}{Louis T. Rader Outstanding Education Undergraduate Student}{University of Virginia Department of Computer Science}{May 2015}
	%  Awarded for my work for several years as a TA for multiple courses in the computer
	% science department at UVA.
	% (\url{https://engineering.virginia.edu/departments/computer-science/news/cs-honors-and-awards#accordion223814})
\end{award}

\section{Peer-reviewed Conference Publications}
\begin{confpub}[in submission]{Counterexample-Guided Repair for Symbolic-Geometric Action Abstractions}{\textbf{Wil Thomason} and Hadas Kress-Gazit}{RSS}{2021}
\end{confpub}
\begin{confpub}[in submission]{Ensuring Progress for Multiple Mobile Robots via Space Partitioning, Motion Rules, and Adaptively Centralized Conflict Resolution}{Claire Liang$^*$, \textbf{Wil Thomason}$^*$, Elizabeth Ricci, and Soham Sankaran}{IROS}{2021}
\end{confpub}
% \begin{confpub}[in submission]{Ensuring Safety and Progress for Independent Multi-Robot Teams in Shared
% 		Space}{\textbf{Wil Thomason}, Claire Liang, Elizabeth Ricci, and Soham Sankaran}{WAFR}{2020}
% 	% We present a framework for guaranteeing safety and progress for all robots operating in a
% 	% 	shared environment containing multiple independent multi-robot teams. Our technique uses a
% 	% 	partition of the freespace into \emph{flow regions} and \emph{open space} to avoid deadlock in
% 	% 	narrow passageways and provide a mechanism for dynamically redistributing robot density in the
% 	% 	environment, and constructs a simple priority-based protocol for requesting access to a spot in
% 	% 	the freespace that guarantees all robots can always eventually reach their goals.
% \end{confpub}

% \begin{confpub}[in submission]{Automatic Distributed Multi-Agent Coordination of Single-Agent
% 		Robot Controllers}{\textbf{Wil Thomason}, Abhishek Anand, Greg Morrisett, Ross Knepper}{IROS}{2020}
% 	% We present a framework for automatically transforming single-agent robot controllers
% 	% 	into distributed multi-agent controllers with guaranteed safety and progress under lightweight
% 	% 	assumptions. The core of this framework is a simple protocol that maintains two invariants:
% 	% 	(a) all executed actions are non-colliding, and (b) there is always a valid plan to reverse
% 	% 	the system back to an earlier state.
% \end{confpub}

\begin{confpub}{A Unified Sampling-Based Approach to Integrated Task and Motion
		Planning}{\\\textbf{Wil Thomason} and Ross Knepper}{ISRR}{2019}
	% We contribute a novel approach to integrated task and motion planning that adapts any
	% 	off-the-shelf motion planning algorithm to automatically solve \emph{task and motion planning
	% 		problems}. We construct a state space embedding the full symbolic and continuous state of a
	% 	planning problem into a single highly factorable space. Combined with a continuous semantics
	% 	for Boolean predicates, which we use to derive navigation functions to task-level salient
	% 	regions of the continuous space, we can efficiently solve complex manipulation problems with
	% 	few domain-specific assumptions.
\end{confpub}

\begin{confpub}{Social Momentum: A Framework for Legible Navigation in Dynamic Multi-Agent
		Environments}{Christoforos Mavrogiannis, \textbf{Wil Thomason}, Ross Knepper}{HRI}{2018}
	% We design a planning framework that aims at generating motion that clearly communicates
	% 	an agent’s intended collision avoidance strategy rather than its destination. Our framework
	% 	estimates the most likely intended avoidance protocols of others based on their past
	% 	behaviors, superimposes them, and generates an expressive and socially compliant robot action
	% 	that reinforces the expectations of others regarding these avoidance protocols.
\end{confpub}

\begin{confpub}{Zero-Shot Learning for Unfamiliar Gesture Recognition}{\textbf{Wil Thomason} and Ross
		Knepper}{ISER}{2016}
	%  We explore a method for achieving increased understanding of complex, situated
	% communications by leveraging coordinated natural language, gesture, and context. These three
	% problems have largely been treated separately, but unified consideration of them can yield
	% gains in comprehension.
\end{confpub}

\section{Journal Publications}
\begin{journpub}[under review]{Social Momentum: Design and Evaluation of a Framework for Socially Competent Robot Navigation}{Christoforos Mavrogiannis, Patr\'{i}cia Alves-Oliveira, \textbf{Wil Thomason}, Ross A. Knepper}{T-HRI}{2021}
\end{journpub}

\begin{journpub}[authors listed in alphabetical order]{An Accurate Real-Time RFID-Based Location System}{Kirti Chawla, Christopher
		McFarland, Gabriel Robins, \textbf{Wil Thomason}}{International Journal of Radio Frequency
		Identification Technology and Applications.}{July 2016}
	% We outline an RFID-based object localization framework addressing challenges of
	% performance and applicability and propose the use of Received Signal Strength (RSS) to model
	% the behavior of radio signals decaying over distance in an orientation-agnostic manner to
	% simultaneously locate a number of stationary and mobile objects.
\end{journpub}

\section{Workshop Presentations}
``Robust, Efficient, and Flexible Robot Planning.'' July 11, 2020.
\emph{RSS Pioneers 2020}
\medskip

``A Flexible Sampling-Based Approach to Task and Motion Planning.'' June 23, 2019.
\emph{RSS 2019 Workshop on Robust Task and Motion Planning}
% \url{http://dyalab.mines.edu/2019/rss-workshop/thomason.pdf}
\medskip

``Which comes first, the task plan or the motion plan?.'' June 30, 2018.
\emph{RSS 2018 Workshop on Exhibition and Benchmarking of Task and Motion Planners}. Joint with Ross
A. Knepper.
% \url{http://www.neil.dantam.name/2018/rss-tmp-workshop/}
\medskip

``Exploiting Heterogeneity in Robot Teams Through a Formalism of Capabilities.'' July
15, 2017. \emph{RSS 2018 Workshop on Heterogeneity and Diversity for Resilience in Multi-Robot Systems}
% \url{https://www.cs.cornell.edu/\textasciitilde wil/papers/rss2017\_workshop\_heterogeneous\_coordination.pdf}
\medskip

``Toward Contextual Grounding of Unfamiliar Gestures for Human-Robot Interaction.''\hfill May 30,
2017. \emph{FG 2017: First International Workshop on Adaptive Shot Learning for Gesture Understanding and
	Production}
% \url{https://www.cs.cornell.edu/\textasciitilde wil/papers/asl4gup2017\_unfamiliargestures.pdf}
\medskip

``Recognizing Unfamiliar Gestures for Human-Robot Interaction through Zero-Shot
Learning.'' June 19th, 2016. \emph{2nd Workshop on Model Learning for Human-Robot Communication, RSS 2016}
% \url{http://www.ece.rochester.edu/projects/rail/mlhrc2016/}

\section{Invited Talks and Consortia}
\begin{symposium}{Search Based Planning Lab}{}{2020\\}
  Invited to present my work on integrated task and motion planning and automatic abstraction repair in the \href{http://www.sbpl.net/}{Search Based Planning Lab}.
\end{symposium}

\begin{symposium}{RSS Pioneers Workshop \emph{(virtual due to COVID-19)}}{}{2020\\}
  Selective annual workshop in conjunction with the Robotics: Science and Systems conference.
  Designed to ``bring together a cohort of the world's top early career researchers to foster
  creativity and collaborations surrounding challenges in all areas of robotics.''
  % \footnote{From the \hyperlink{https://sites.google.com/view/rsspioneers2020\#h.p\_uRuKS3yP_C3Y}{RSS Pioneers 2020 website}.}
  (\emph{33.7\% acceptance rate})
\end{symposium}

\section{Teaching Experience}

\begin{class}{CS 4750}{Foundations of Robotics}{Cornell University}{Fall 2016 \& Fall 2017}
	\item Graduate TA (syllabus creation, coding project creation and implementation, grading, office hours,
	occasional lecturing). Senior and graduate-level elective.
\end{class}

\begin{class}{CS 1110}{Introduction to Computing Using Python}{Cornell University}{Fall 2015}
	\item Head graduate TA (coordinating staff, giving review lectures, supervising lab sessions, grading,
	office hours). Introductory undergraduate CS course.
\end{class}

\begin{class}{ENG 1501}{Introduction to Aerial Robotics}{University of Virginia}{Fall 2014}
	\item Instructor. Designed and taught 1-credit special-topics undergraduate elective introducing
	core topics in robotics. Students built and programmed their own quadrotor robots and learned
	about basic kinematics, control, and perception.
\end{class}

\begin{class}{CS 4610}{Programming Languages}{University of Virginia}{Spring 2015}
	\item Undergraduate TA. Senior-level elective.
\end{class}

\begin{class}{CS 4710}{Artificial Intelligence}{University of Virginia}{Spring 2015}
	\item Undergraduate TA. Senior-level elective.
\end{class}

\begin{class}{CS 4414}{Operating Systems}{University of Virginia}{Spring 2014}
	\item Undergraduate TA (office hours, assignment creation). Senior-level core course.
\end{class}

% Gross abuse of date field here because of weird line flow with long sequence of dates
\begin{class}{CS 2150}{Program and Data Representation}{University of Virginia}{(Fall 2013 -- Spring 2015). Undergraduate TA (office hours, lab supervision, grading). Sophomore-level core course.}
\end{class}

\section{Outreach}
\begin{loneboldlist}
	\item[Reviewer for \hyperlink{https://blackinai.github.io/}{Black in AI}:] Reviewed abstracts for
	BAI workshop.\hfill2017--2020
	\item[Mentor for \hyperlink{https://blackinai.github.io/}{Black in AI}:] Advised mentee on
	Ph.D. application process.\hfill2019--2020
	\item[\href{https://www.eyh.cornell.edu}{Expanding Your Horizons}:] Workshop
	Organizer/Leader.\hfill Spring 2016, 2017, 2018
	% \begin{itemize}
	% 	\setlength\itemsep{0em}
	% 	\item Created interface for introducing the concept of programming with state machines
	% 	\item Wrote ROS software to run state machine code on a Kuka youBot
	% \end{itemize}
	\item[\href{http://acm.cs.virginia.edu/hspc.php}{UVa HS Programming Contest}:]
	Organizer/volunteer.~\hfill Spring 2014, 2015
	% \begin{itemize}
	% 	\setlength\itemsep{0em}
	% 	\item Created contest problems
	% 	\item Planned contest logistics
	% 	\item Helped run contest
	% \end{itemize}
	\item[UVa CS Education Week] Ran intro CS workshop.\hfill Winter 2014, 2015
	% \\Visited area schools to teach workshops on introductory programming with JKarel.
\end{loneboldlist}\medskip

\section{Service}
\begin{loneboldlist}
  \item[Faculty chair:] \href{https://sites.google.com/view/rsspioneers2021}{RSS Pioneers 2021} workshop.
  \item[Reviewer:] ICRA (2016, 2019--2021), IROS (2019, 2021), RSS (2019), WAFR (2018), MRS (2019), RO-MAN (2016), RA-L (2021), IJCAI (2021), AURO, T-ASE (2020), and SIMPAR (2018).
  \item[Departmental Service:] Student representative to Diversity and Inclusion Committee (2020--2021), Colloquim Czar (2016--2020), Administrative Colloquium Czar (2016--2019), Ph.D. Mentor Czar (2016--2018).
\end{loneboldlist}

% \section{Mentoring}

% \section{Open-Source Software}

% \section{Invited Talks}
\section{Professional Experience}
\begin{job}{Graduate Research Assistant}{January 2020 --
		Present}{\href{http://verifiablerobotics.com}{VRRG}, Department of Computer Science, Cornell University}
\end{job}

\begin{job}{Graduate Research Assistant}{August 2015 -- December 2019}{Robotic Personal Assistants
		Lab, Department of Computer
		Science, Cornell University}
	% \item Work with Professor Ross Knepper on problems in the domains of integrated
	% task and motion planning, human-robot interaction and multi-agent planning.
	% See publications co-authored with Professor Knepper.
\end{job}

\begin{job}{Software Engineering Intern}{May 2015 -- August 2015}{Fluencia, Alexandria,
		VA}
	Worked on adding voice recognition for speech practice exercises.
	% \begin{itemize}
	% 	\setlength\itemsep{0em}
	% 	\item Improved voice understanding software
	% 	\item Integrated into language learning website
	% 	\item Created test and development framework for voice exercises
	% \end{itemize}
\end{job}

\begin{job}{Undergraduate Research Assistant}{August 2014 -- July 2015}{Department of Computer
		Science, The University of Virginia}
	Work with Professor Westley Weimer on automatic software functionality transplantation.
	% We developed an algorithm based on analyzing differences in test suite
	% performance to identify software modules responsible for specific functionality and a
	% combination of function dependency tracing and extraction with the GenProg program repair tool
	% to perform transplants.
\end{job}

\begin{job}{Software Development Engineer Intern}{May 2014 -- August 2014}{Accounts Client Team,
		Microsoft, Redmond, WA}
	Implemented cryptographic operations and network protocol for passwordless login feature in
	Microsoft Accounts Android app.
	% \begin{itemize}
	% 	\setlength\itemsep{0em}
	% 	\item Implemented and tested cryptographic operations and network protocol for forthcoming
	% 	      feature in Microsoft Accounts Android app. The implemented feature enables more convenient and
	% 	      more secure sign-in for Microsoft accounts via mobile devices.
	% 	\item Coordinated the work of several other interns working on related components of the same
	% 	      project to ensure successful integration and delivery of the fully-functional feature ahead of
	% 	      schedule.
	% \end{itemize}
\end{job}

\begin{job}{Software Development Engineer Intern}{May 2013 -- August 2013}{Xbox LIVE Cloud Security
		Team, Microsoft, Redmond, WA}
	Designed and implemented a service for real-time logging and auditing of security records in
	Xbox LIVE. Initiated and completed a rewrite of an internal library to improve performance and
	provide a better API.
	% \begin{itemize}
	% 	\setlength\itemsep{0em}
	% 	\item Designed, implemented, and shipped a RESTful web service capable of logging and auditing
	% 	      security records in Xbox LIVE in real time. Also designed, created, and tuned associated
	% 	      database and procedures. Service is currently in use in the Xbox LIVE network.
	% 	\item Spearheaded and completed a total rewrite of an important development library – starting
	% 	      with old, cobbled together library used across principal components of the Xbox LIVE network,
	% 	      redesigned and reimplemented the entire library to provide a faster and easier to use
	% 	      interface to the same core functionality.
	% \end{itemize}
\end{job}

\begin{job}{Undergraduate Research Assistant}{January 2013 -- May 2014}{Department of Computer
		Science, The University of Virginia}
	Work with Professor Gabriel Robins on real-time localization of objects using passive RFID tags.
	% See publication ``An Accurate Real-Time RFID-Based Location System''.
\end{job}


\section{Technical\\Skills}
\begin{loneboldlist}
	\item[Programming Languages:] Python, C++, Julia, Rust, Lua, C, Haskell, OCaml, etc.
	\item[Technologies:] Linux, ROS, OMPL, Jax, PyTorch, Git, CUDA, etc.
\end{loneboldlist}

% \section{Languages}
%
% English \emph{(Fluency: Native speaker)}\\
% German \emph{(Fluency: Working)}
\end{document}
